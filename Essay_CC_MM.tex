% Options for packages loaded elsewhere
\PassOptionsToPackage{unicode}{hyperref}
\PassOptionsToPackage{hyphens}{url}
\PassOptionsToPackage{dvipsnames,svgnames*,x11names*}{xcolor}
%
\documentclass[
]{article}
\usepackage{lmodern}
\usepackage{amssymb,amsmath}
\usepackage{ifxetex,ifluatex}
\ifnum 0\ifxetex 1\fi\ifluatex 1\fi=0 % if pdftex
  \usepackage[T1]{fontenc}
  \usepackage[utf8]{inputenc}
  \usepackage{textcomp} % provide euro and other symbols
\else % if luatex or xetex
  \usepackage{unicode-math}
  \defaultfontfeatures{Scale=MatchLowercase}
  \defaultfontfeatures[\rmfamily]{Ligatures=TeX,Scale=1}
\fi
% Use upquote if available, for straight quotes in verbatim environments
\IfFileExists{upquote.sty}{\usepackage{upquote}}{}
\IfFileExists{microtype.sty}{% use microtype if available
  \usepackage[]{microtype}
  \UseMicrotypeSet[protrusion]{basicmath} % disable protrusion for tt fonts
}{}
\usepackage{xcolor}
\IfFileExists{xurl.sty}{\usepackage{xurl}}{} % add URL line breaks if available
\IfFileExists{bookmark.sty}{\usepackage{bookmark}}{\usepackage{hyperref}}
\hypersetup{
  colorlinks=true,
  linkcolor=blue,
  filecolor=Maroon,
  citecolor=Blue,
  urlcolor=Blue,
  pdfcreator={LaTeX via pandoc}}
\urlstyle{same} % disable monospaced font for URLs
\usepackage[margin=1in]{geometry}
\usepackage{graphicx}
\makeatletter
\def\maxwidth{\ifdim\Gin@nat@width>\linewidth\linewidth\else\Gin@nat@width\fi}
\def\maxheight{\ifdim\Gin@nat@height>\textheight\textheight\else\Gin@nat@height\fi}
\makeatother
% Scale images if necessary, so that they will not overflow the page
% margins by default, and it is still possible to overwrite the defaults
% using explicit options in \includegraphics[width, height, ...]{}
\setkeys{Gin}{width=\maxwidth,height=\maxheight,keepaspectratio}
% Set default figure placement to htbp
\makeatletter
\def\fps@figure{htbp}
\makeatother
\setlength{\emergencystretch}{3em} % prevent overfull lines
\providecommand{\tightlist}{%
  \setlength{\itemsep}{0pt}\setlength{\parskip}{0pt}}
\setcounter{secnumdepth}{-\maxdimen} % remove section numbering
\usepackage{fancyhdr}
\pagestyle{fancy}
\fancyhf{}
\rhead[\rightmark]{Ensayo I Geosistemas. M. Mardones. II Sem, 2021}
\lfoot[\thepage]{}
\rfoot[]{\thepage}
\newlength{\cslhangindent}
\setlength{\cslhangindent}{1.5em}
\newenvironment{cslreferences}%
  {\setlength{\parindent}{0pt}%
  \everypar{\setlength{\hangindent}{\cslhangindent}}\ignorespaces}%
  {\par}

\title{\includegraphics[width=4cm,height=\textheight]{logoUMAG.jpg}}
\author{}
\date{\vspace{-2.5em}}

\begin{document}
\maketitle


\pagenumbering{gobble}

%\begin{titlepage}
\begin{flushleft}
\Large{\textbf{Ensayo Científico I}}\\
\vspace*{2\baselineskip}
\LARGE{\textbf{Fenómenos atmosféricos regionales y sinópticos y su influencia en la formación hielo marino en la Península Antártica}}\\
\vspace*{3\baselineskip}
\Large{Curso GeoSistemas}\\
\vspace*{1\baselineskip}
\Large{Semestre-2 2021 }\\
\vspace*{3\baselineskip}
\end{flushleft}
\begin{flushright}
\large{\textbf{Mauricio Mardones Inostroza}}\\
\large{\textbf{Biólogo Marino}}\\
\vspace*{2\baselineskip}
\normalsize{Alumno Programa de Doctorado Ciencias Antárticas y Sub-Antárticas}\\
\vspace*{1\baselineskip}
\normalsize{Universidad de Magallanes, Chile}\\
\vspace*{1\baselineskip}
\normalsize{\textbf{Profesores}}\\
Dr. Ricardo De Pol-Holz\\
\vspace*{1\baselineskip}
\normalsize{\textbf{Fecha}}\\
Noviembre, 2021
\end{flushright}

% \end{titlepage}


\hypersetup{linkcolor = black}
\newpage
\pagenumbering{roman}
%\tableofcontents
%\addcontentsline{toc}{section}{\contentsname}

\newpage



\pagenumbering{arabic}
\hypersetup{linkcolor = blue}

\fontsize{12}{26}
\selectfont{}

\hypertarget{introduccion}{%
\subsection{1. INTRODUCCION}\label{introduccion}}

La Antartica y el Océano Austral (OA) que lo circunda son componentes
activos del clima global. Estos dos elementos también son profundamente
influenciados por las muchas variabilidades atmosféricas remotas y
regionales en períodos que van desde escalas de tiempo sinópticas a
geológicas (Yuan, \protect\hyperlink{ref-Yuan2004}{2004}). En este
sentido, la Antártica y el Oceáno Austral asociado han jugado un rol
fundamental para entender la dinámica de muchos forzantes ambientales,
entre ellos, el hielo marino, y su variabilidad asociada a los distintos
fenómenos atmosféricos que influyen en estos procesos. Este ensayo tiene
como finalidad identificar y describir las particularidades de los
principales fenómenos atmosféricos y como estos influyen en la
formación, expansión y contracción de hielo Antártico. A su vez daremos
una mirada historica y también como el Cambio Climático (CC) genera
desacoples en estos fenómenos atmosféricos, y con ello, sus implicancias
en la dinámica propia del hielo marino Antártico. Todo esto con un foco
en particular en la Península Antártica (PA). Por último, trataremos de
entender como el hielo marino y su ciclos se correlacionan con la
dinámica de las poblaciones marinas del krill (\emph{Euphausia
superba}), las cuales son la base de los ecosistemas marinos en el
Océano Austral.

\pagebreak

\hypertarget{desarrollo}{%
\subsection{2. DESARROLLO}\label{desarrollo}}

La superficie del océano alrededor de la Antártica se congela en
invierno y se derrite cada verano. El hielo marino antártico
generalmente alcanza su extensión máxima anual a mediados o finales de
septiembre, y alcanza su mínimo anual a finales de febrero o principios
de marzo (Scott, \protect\hyperlink{ref-Scott2019}{2019}) (Figura 1). El
hielo marino antártico se extiende a unos 12 millones de kilómetros
cuadradas en invierno convirtiendola en un continente rodeado por un
vasto océano lo cual tiene muchas implicancias en como aquí se
manifiestan los procesos climáticos-atmosféricos.

\begin{figure}

{\centering \includegraphics[width=0.7\linewidth]{images/Fig4} 

}

\caption{Extensión del hielo Antártico y su fluctuación intra e interanual (ex. www.climate.gov)}\label{fig:unnamed-chunk-1}
\end{figure}

El hielo marino puede expandirse libremente a través del Océano Austral
en invierno, pero no puede acercarse al Polo Sur de lo que la costa
antártica lo permitirá. En promedio, alrededor del 15 por ciento de la
cubierta de hielo invernal del Océano Austral, es por ello que la mayor
parte del hielo marino de la Antártica tiene solo un invierno como
máximo. Como resultado, el hielo marino antártico es relativamente
delgado, a menudo de 1 metro o menos. Sin embargo, las nevadas a menudo
espesan el hielo marino antártico. La pesada carga de nieve puede
deprimir los témpanos de hielo, y el agua de mar puede inundar
posteriormente esos témpanos (Scott,
\protect\hyperlink{ref-Scott2019}{2019}).

Con respecto al hielo continental, la elevación antártica promedia cerca
de 2000 metros (más de 1000 mm. sobre el nivel del mar). Las partes más
altas de la capa de hielo, cerca del centro de la Antártida Oriental,
rivalizan con la altura de sus montañas más altas, a casi 4000 metros.
Por todas estas características, este océano es una de los áreas más
sensibles al CC. Los forzantes claves y su influencia en procesos claves
dentro del OA ya han experimentado cambios en sus dinámicas naturales,
entre ellos se pueden identificar; la temperatura del océano, la
dinámica del hielo marino, la estratificación, las corrientes, entre
otros (Morley et al., \protect\hyperlink{ref-Morley2020}{2020};
Sylvester et al., \protect\hyperlink{ref-Sylvester2021}{2021}).

En comparación con el Ártico, el continente Antártico está respondiendo
con menos intensidad al cambio climático (Groh \& Horwath,
\protect\hyperlink{ref-Groh2021}{2021}). En este sentido, el OA está
amortiguando los impactos de vientos y patrones climáticos que tienden a
aislarla de lo que podrían influir las entradas de aire cálido. Sin
embargo, y en promedio, la Antártica se ha descongelado y ha perdido
mucho hielo (Nicolas \& Bromwich,
\protect\hyperlink{ref-Nicolas2014}{2014}), y esto puede tener
implicancias futuras. El reporte del Grupo Intergubernamental de
Expertos sobre el CC (IPCC por sus siglas en inglés) (IPCC,
\protect\hyperlink{ref-IPCC2014}{2014}) encontró una tendencia al
calentamiento en la Antártica, pero no a niveles concluyentes. Por otro
lado, a finales del siglo XX, el agujero de ozono y sus efectos en la
circulación del aire pueden haber protegido parcialmente al continente
de la influencia del calentamiento global de las emisiones de gases de
efecto invernadero (Damiani et al.,
\protect\hyperlink{ref-Damiani2020}{2020}). Respecto a la influencia del
clima sobre la dinámica del hielo marino en la PA, Massom et al.
(\protect\hyperlink{ref-Massom2008}{2008}) identificaron una extrema
fase de compactación de hielo frente a anomalías de temperatura, vientos
y diferencias de presión atmosferica. La variabilidad y ciclos del hielo
internanuales e históricos son influenciados por forzantes atmosfericos.
Estos procesos estan profusamente descritas en la literatura (Carrasco,
\protect\hyperlink{ref-Carrasco2018}{2018}; Damiani et al.,
\protect\hyperlink{ref-Damiani2020}{2020}; Massom et al.,
\protect\hyperlink{ref-Massom2008}{2008}; Mayewski et al.,
\protect\hyperlink{ref-Mayewski2017}{2017}; Rohli \& Vega,
\protect\hyperlink{ref-Rohli2018}{2018}; S. E. Stammerjohn et al.,
\protect\hyperlink{ref-Stammerjohn2008}{2008}; Stammerjohn et al.,
\protect\hyperlink{ref-Stammerjohn2008a}{2008}; Turner,
\protect\hyperlink{ref-Turner2004}{2004}; Yuan,
\protect\hyperlink{ref-Yuan2004}{2004}).

Sin embargo, y al igual que otros continentes, la Antártica siente los
impactos del CC a diferentes ritmos en diferentes lugares. En la
Península Antártica, de menor elevación y con menor latitud, experimentó
rápidos aumentos de temperatura durante varias décadas en el siglo XX.
Luego, a finales de la década de 1990, los aumentos de temperatura se
ralentizaron, y desde 2000, la Península se ha enfriado. Durante los
últimos 40 años la PA ha sido una de las zonas más afectadas de la
región en términos de cambios ambientales. Los más notables de estos
tienen relación con un aumento de los vientos del oeste, un aire más
alto y profundo temperaturas del océano, y reducciones en el hielo
marino, tanto su extención y el tiempo de formación. Estudios del 2016 y
2021 concluyeron que el reciente enfriamiento se puede explicar por la
variabilidad natural y condiciones atmosféricas del Océano Pacífico. A
pesar del reciente enfriamiento o \emph{hiatus} (Carrasco et al.,
\protect\hyperlink{ref-Carrasco2021b}{2021}; Turner et al.,
\protect\hyperlink{ref-Turner2016}{2016}), las partes septentrionales de
la PA siguen siendo mucho más cálidas de lo que eran a mediados del
siglo pasado (Nicolas \& Bromwich,
\protect\hyperlink{ref-Nicolas2014}{2014}; Scott,
\protect\hyperlink{ref-Scott2020}{2020}; Turner et al.,
\protect\hyperlink{ref-Turner2005}{2005}) (Figura 2), considerando a la
PA como una de las partes de la Tierra que ha experimentado un gran
calentamiento los últimos 50 años, habiendo aumentado a una tasa de 0.56
ºC década (1 durante el año y 1.09 ºC década) (Turner et al.,
\protect\hyperlink{ref-Turner2005}{2005}).

\pagebreak

\begin{figure}

{\centering \includegraphics[width=1\linewidth]{images/Fig3} 

}

\caption{Tendencias de temperatura durante todo el año 1958 al 2012 por década en la Antártica (Ex. Nicolas y Bromwich, 2014}\label{fig:unnamed-chunk-2}
\end{figure}

En relación a variabilidad historica de la formación de hielo marino en
el OA, es necesario entender como ha sido la producción en el pasado y
ver como se están produciendo los cambios, tanto así en la actualidad
como en el futuro por influencia y desajustes de las condiciones
atmosféricas imperantes. En función de las características descritas
previamente en relación a la dinámica del hielo marino en el continente
Antártico, es necesario entender entonces cuales son los forzantes
atmosféricos que influyen en ello. Un corolario de esto es la evidente
demostración de forzantes globales que afectan a la dinámica del hielo
en el Océano Austral y en particular en la Península Antártica, que se
hacen sentir en el presente y también afectarán en el futuro.

Con estos antecedentes comprender la dinamica del hielo marino en la PA
y su relación con los fenómenos atmosfericos que influyen en ello es el
objetivo de este ensayo. Las forzantes atmosférico-oceanográficos son
los que se manifiestan por cambios producidos en la atmósfera y que
tienen implicancias en el mar. Entre aquellos forzantes, los que han
sido mas estudiados e identifcados como factores influyentes en
componentes del OA, son el Southern Annular Mode (SAM) y El Niño
Southern Oscillation (ENSO).

\hypertarget{enso-y-su-influencia-en-el-hielo-marino-en-la-penuxednsula-antuxe1rtica}{%
\subsection{2.1. ENSO y su influencia en el hielo marino en la Península
Antártica}\label{enso-y-su-influencia-en-el-hielo-marino-en-la-penuxednsula-antuxe1rtica}}

Un fenómeno oceanográfico-atmosférico que tiene implicancias en los
océanos es la Oscilación del Sur El Niño, que en inglés es El Niño
Southern Oscilation (de aquí en más ENSO, por sus siglas en inglés). El
ENOS, que incluye su contraparte La Niña, es la variabilidad climática
interanual más significativa que tiene lugar en el planeta. Ocurre en la
cuenca ecuatorial del Océano Pacifico con alteraciones de presión
atmosféricas, cuyos efectos se hacen sentir más allá de los trópicos, en
diversos lugares alrededor del planeta, incluso en la Antártica
(Carrasco, \protect\hyperlink{ref-Carrasco2018}{2018}; S. E. Stammerjohn
et al., \protect\hyperlink{ref-Stammerjohn2008}{2008}), indicando una
relación inversa entre las anomalías estaciónales de presión al nivel
del mar en las latitudes polares y ENSO con desfase negativo de un año
(Carrasco, \protect\hyperlink{ref-Carrasco2018}{2018}). Un evento ``El
Niño'' consiste principalmente en un calentamiento por sobre lo normal
de la superficie del Océano Pacífico ecuatorial central y oriental
(mayor a 25ºC), acompañado por un aumento del nivel del mar (mayor a 15
cm.). En cambio durante el fenómeno de El Niño los vientos alisios se
debilitan o dejan de soplar, la máxima temperatura marina se desplaza
hacia la Corriente de Perú que es relativamente fría y la mínima
temperatura marina se desplaza hacia el Sureste Asiático. Esto provoca
el aumento de la presión atmosférica en el sureste asiático y la
disminución en América del Sur. Todo este cambio ocurre en un intervalo
de seis meses, aproximadamente desde junio a noviembre.

Luego de una revisión de la influencia de este forzante, Carrasco
(\protect\hyperlink{ref-Carrasco2018}{2018}) indica que existe una
relación entre la Oscilación del Sur (OS) y la concentración del hielo
marino que rodea la Antártica, en particular en el sector del Mar de
Weddell por medio del Dipolo Antártico (ADP), el cual es una
variabilidad climática de latitudes altas en el sistema interactivo de
la atmósfera-océano-hielo de fase opuesta entre el hielo marino y las
anomalías de la temperatura superficial del mar en el Océano Pacífico
Sur y el Océano Atlántico Sur, y que responde a la forzante del ENOS
(Yuan, \protect\hyperlink{ref-Yuan2004}{2004}). Carrasco
(\protect\hyperlink{ref-Carrasco2018}{2018}) analizó el comportamiento
clímatico del sector norte de la PA durante los mayores eventos del ENSO
(Figura 3) que sucedieron en los ultimos 50 años, y en donde concluyeron
que existe un impacto de ENOS en la formación del hielo marino en la
Antártida, en particular en el sector oeste de la PA, en el Mar de
Bellingshausen y Amundsen. La formación y extensión hacia el norte del
hielo marino puede verse frenada cuando prevalecen vientos del noroeste
en el sector, situación que se da con eventos La Niña. Por el contrario,
la formación y desplazamiento del hielo marino puede verse favorecido
cuando la frecuencia de vientos del noroeste es baja y débil o incluso
la circulación del viento sea del oeste y suroeste, situación que ocurre
en los eventos El Niño. Así, una baja presión en el Mar de
Bellingshausen asociado a El Niño puede favorecer un flujo con
componente del sur (o flujo del noroeste menos frecuente y/o más
débiles) en el sector del Mar de Bellingshausen y un flujo del sur más
débil en el sector del Mar de Ross, favoreciendo y desfavoreciendo,
respectivamente, la formación y extensión de hielo marino hacia el norte
{[}Carrasco2021a{]}.

\begin{figure}

{\centering \includegraphics[width=0.8\linewidth]{images/Fig5} 

}

\caption{Comportamiento del Índice de Oscilación del Sur (IOS, eje izquierdo) junto a las anomalías de la concentración de hielo marino (CHM) en el sector norte del Mar de Bellingshausen. (Ex. Carrasco, 2018)}\label{fig:unnamed-chunk-3}
\end{figure}

\pagebreak

\hypertarget{sam-y-su-influencia-en-el-hielo-marino-en-la-penuxednsula-antuxe1rtica}{%
\subsection{2.2. SAM y su influencia en el hielo marino en la Península
Antártica}\label{sam-y-su-influencia-en-el-hielo-marino-en-la-penuxednsula-antuxe1rtica}}

La variabilidad de la circulación atmosférica estacional en el
hemisferio sur está modulado por varios modos climáticos de variabilidad
de la circulación atmosférica. Uno de los modos que dominan esta
variabilidad es el llamado Modo Anular del SUR o Southern Annular Mode
(de aquí en mas SAM, por sus siglas en inglés). En general, la
variabilidad SAM se caracteriza por anomalías zonales simétricas de
presión atmosférica de opuestos signo entre la Antártida y las latitudes
medias. Varios estudios han explorado las posibles conexiones entre
hielo marino del Océano Austral y SAM (Massom et al.,
\protect\hyperlink{ref-Massom2008}{2008}; Nicolas \& Bromwich,
\protect\hyperlink{ref-Nicolas2014}{2014}; S. E. Stammerjohn et al.,
\protect\hyperlink{ref-Stammerjohn2008}{2008}). Por otro lado, el
agotamiento del ozono influye directamente en el modo principal de
variabilidad de la circulación atmosférica en las regiones
extratropicales del sur. Se cree que la tendencia del verano (intensidad
SAM +) se debe principalmente al agotamiento del ozono polar
estratosférico (Damiani et al.,
\protect\hyperlink{ref-Damiani2020}{2020}).

Una forma de medir el SAM es a través de un índice calculado como el
gradiente de presión entre las latitudes medias y la Antártida, que
cuando es muy positivo, da como resultado vientos del oeste que son más
fuertes que el promedio y se desplazan hacia el polo. Desde 1957 ha
habido un aumento significativo de la fase SAM positivo en el verano y
otoño austral. La variabilidad del SAM tiene un impacto significativo en
la temperatura de la superficie antártica, la precipitación y el hielo
marino (Marshall et al., \protect\hyperlink{ref-Marshall2017}{2017}).

\pagebreak

Al mismo tiempo, la variabilidad del hielo marino observada en las
últimas dos décadas de observaciones satelitales muestran disminuciones
en concentración y duración en el oeste de la Península Antártica
(Newman et al., \protect\hyperlink{ref-Newman2016}{2016}). Estudios
previos han demostrado que la duración y la concentración mensual del
hielo marino Antártico han disminuido fuertemente en la región oeste de
la PA. Estas observaciones sugieren que las tendencias del hielo marino
en la región oeste de la PA son principalmente reflejando los cambios en
las interacciones hielo-atmósfera que ocurren durante la primavera-otoño
durante la retirada del hielo marino y la posterior avance por efectos
de las fases y duraciones del SAM (Figura 4) (S. E. Stammerjohn et al.,
\protect\hyperlink{ref-Stammerjohn2008}{2008}; Stammerjohn et al.,
\protect\hyperlink{ref-Stammerjohn2008a}{2008}).

\begin{figure}

{\centering \includegraphics[width=0.6\linewidth]{images/Fig6} 

}

\caption{Representación esquemática de la respuesta de hielo en latitudes altas por efecto de la fase SAM + (Ex. Stammerjohn et al. 2008)}\label{fig:unnamed-chunk-4}
\end{figure}

Finalmente es necesario también señalar que ambos fenomenos
atmosfericos, es decir, el Southern Annular Mode (SAM) y El Niño
Southern Oscillation (ENSO) están interanctuando constantemente y tienen
efectos combinados sobre la criósfera. En este sentido existen estudios
recientes que han examinado los impactos combinados de ENSO y SAM en la
respuesta de alta latitud tanto de la circulación atmosférica y hielo
marino (S. E. Stammerjohn et al.,
\protect\hyperlink{ref-Stammerjohn2008}{2008}; Stammerjohn et al.,
\protect\hyperlink{ref-Stammerjohn2008a}{2008}).

Respecto a antecedentes históricos de la dinámica del hielo marino en la
PA y sus condicionantes atmosféricas, Vorrath et al.
(\protect\hyperlink{ref-Vorrath2020}{2020}) analizaron imágenes
satelitales del hielo marino durante los últimos 40 años y observaron
diiferencias entre las estimaciones del hielo marino basadas en
biomarcadores, el salida del modelo de los últimos 240 años, registros
de núcleos de hielo y reconstruidos patrones de circulación atmosférica
como el Niño -- Oscilación del Sur (ENSO) y Anular del Sur Modo (SAM)
(Figura 5). En este trabajo identificaron los distintos patrones
atmosféricos y los efectos diferenciados sobre la formación de hielo en
la PA. Estos hallazgos tienen implicancias importantes frente a
escenarios climáticos futuros y su impacto en este importante compomente
de la criósfera.

\begin{figure}

{\centering \includegraphics[width=1\linewidth]{images/Fig7} 

}

\caption{Ilustración de momentos historicos y los impulsores ambientales en la formación de hielo en la PA (Ex. Vorrath et al. 2020)}\label{fig:unnamed-chunk-5}
\end{figure}

\pagebreak

Los fenómenos atmosféricos y su conectividad con la dinámica del hielo
marino en la Península Antártica no solo tiene implicancias en
regulaciones climáticas si no también en el ecosistema del OA asociado.
Existen contundentes demostraciones del estrecho vínculo entre los
ciclos de formación de hielo marino antártico y la dinámica de las
poblaciones de krill, en particular en períodos de reproducción y
crecimiento (H. Flores et al., \protect\hyperlink{ref-Flores2012}{2012};
Hauke Flores et al., \protect\hyperlink{ref-Flores2012a}{2012}; Piñones,
\protect\hyperlink{ref-ILAIA2021}{2021}; Piñones \& Fedorov,
\protect\hyperlink{ref-Pinones2016}{2016}; Walsh et al.,
\protect\hyperlink{ref-Walsh2020}{2020}). En este sentido, el krill
depende de las condiciones ambientales de temperatura, clorofila pero
por sobre todo del hielo marino para su supervivencia, ya que utiliza al
hielo marino como refugio y este a su vez se convierte en una zona de
retención del alimento, como lo indica la Figura 6.

\begin{figure}

{\centering \includegraphics[width=0.9\linewidth]{images/Fig10} 

}

\caption{Esquema del ciclo de vida del krill y su relación con la formación de hielo marino Antárico (Ex. Piñones y Fedorov, 2016)}\label{fig:unnamed-chunk-6}
\end{figure}

Dado que durante las últimas cuatro décadas la PA ha sido una de las
zonas más afectadas de la región en términos de cambios ambientales. Los
más notables de estos generan un aumento de los vientos del oeste, un
aire más alto y profundo temperaturas del océano, y esto se transfiere
en reducciones en el hielo marino, tanto su extensión y el tiempo de
formación. Desde el comienzo de los registros de hielo basados en
satélites cubierta (década de 1980) se ha observado que la formación de
hielo ahora ocurre casi dos meses después que hace 30 años, y esto puede
tener implicaciones para la supervivencia de las larvas de krill .Esta
situación esta comenzando a preocupar a diversos investigadores y
administradores por el futuro de esta importante población base del
ecosistema antartico en términos de trama trófica asociada, pero tambien
en su impacto en las pesquerías que explotan el recurso.

\pagebreak

\hypertarget{discusion}{%
\subsection{3. DISCUSION}\label{discusion}}

Como síntesis de lo anteriormente expuesto, existen fuertes evidencias
de la interacción atmosfera-hielo marino y sus consecuencias comprobadas
en el continente Atrártico y en particular en la PA, las cuales son
consistentes con el rápido calentamiento de la región. Los cambios en la
duración de la temporada de hielo se deben principalmente por cambios en
el avance del hielo marino otoñal y en menor grado el retiro del hielo
marino de primavera. Las mayores anomalías hacia un un retroceso algo
más temprano y un avance mucho más tarde co-ocurren con fuertes vientos
del norte durante La Niña y / o + SAM y una combinación de estos.

Mas allá de lo técnico y complejo que es recabar conocimiento y
entendimiento sobre como los fenómenos atmosféricos como el ENSO y el
SAM se conectan con las distintas componentes de la tierra, y en
particular con el continente antártico, este ensayo científico fue un
paso inicial y un sustrato cognitivo para, en primer lugar, reconocer la
importancia de estos fenómenos, como ellos se desencadenan y como
actuan, y a su vez identificar las fuertes implicancias que ello tiene
para otros procesos asociados, impactando tambien a los ecosistemas que
componen a la biósfera.Es evidente que, si bien son fenómenos
atmósfericos que estan en funcionamiento hace miles de años, cualquier
forzante antropogénica puede modificar sus ciclos, dinamica temporal así
como su magnitud y con ello generar desequilibrios indeterminados en los
productivos pero frágiles ecosistemas asociados.

\pagebreak

\hypertarget{referencias}{%
\subsection*{4. REFERENCIAS}\label{referencias}}
\addcontentsline{toc}{subsection}{4. REFERENCIAS}

\hypertarget{refs}{}
\begin{cslreferences}
\leavevmode\hypertarget{ref-Carrasco2018}{}%
Carrasco, J. F. (2018). Señales atmosféricas y del hielo marino
asociadas a ENOS en el sector norte de la península antártica.
\emph{Anales Del Instituto de La Patagonia}, \emph{46}(1), 33--47.
\url{https://doi.org/10.4067/s0718-686x2018000100033}

\leavevmode\hypertarget{ref-Carrasco2021b}{}%
Carrasco, J. F., Bozkurt, D., \& Cordero, R. R. (2021). A review of the
observed air temperature in the Antarctic Peninsula. Did the warming
trend come back after the early 21st hiatus? \emph{Polar Science},
\emph{28}(February), 100653.
\url{https://doi.org/10.1016/j.polar.2021.100653}

\leavevmode\hypertarget{ref-Damiani2020}{}%
Damiani, A., Cordero, R. R., Llanillo, P. J., Feron, S., Boisier, J. P.,
Garreaud, R., Rondanelli, R., Irie, H., \& Watanabe, S. (2020).
Connection between antarctic ozone and climate: Interannual
precipitation changes in the Southern Hemisphere. \emph{Atmosphere},
\emph{11}(6), 1--19. \url{https://doi.org/10.3390/atmos11060579}

\leavevmode\hypertarget{ref-Flores2012}{}%
Flores, H., Atkinson, A., Kawaguchi, S., Krafft, B. A., Milinevsky, G.,
Nicol, S., Reiss, C., Tarling, G. A., Werner, R., Bravo Rebolledo, E.,
Cirelli, V., Cuzin-Roudy, J., Fielding, S., Groeneveld, J. J.,
Haraldsson, M., Lombana, A., Marschoff, E., Meyer, B., Pakhomov, E. A.,
\ldots{} Werner, T. (2012). Impact of climate change on Antarctic krill.
\emph{Marine Ecology Progress Series}, \emph{458}, 1--19.
\url{https://doi.org/10.3354/meps09831}

\leavevmode\hypertarget{ref-Flores2012a}{}%
Flores, H., Franeker, J. A. van, Siegel, V., Haraldsson, M., Strass, V.,
Meesters, E. H., Bathmann, U., \& Wolff, W. J. (2012). The association
of Antarctic krill Euphausia superba with the under-ice habitat.
\emph{PLoS ONE}, \emph{7}(2).
\url{https://doi.org/10.1371/journal.pone.0031775}

\leavevmode\hypertarget{ref-Groh2021}{}%
Groh, A., \& Horwath, M. (2021). Antarctic ice mass change products from
GRACE/GRACE-fo using tailored sensitivity kernels. \emph{Remote
Sensing}, \emph{13}(9). \url{https://doi.org/10.3390/rs13091736}

\leavevmode\hypertarget{ref-IPCC2014}{}%
IPCC. (2014). \emph{Climate Change 2014: Synthesis Report. Contribution}
(p. 169).

\leavevmode\hypertarget{ref-Marshall2017}{}%
Marshall, G. J., Thompson, D. W. J., \& Broeke, M. R. (2017).
Circulation Patterns in Antarctic Precipitation. \emph{Geophysical
Research Letters RESEARCH RESEARCH}, \emph{44}, 11580--11589.

\leavevmode\hypertarget{ref-Massom2008}{}%
Massom, R. A., Stammerjohn, S. E., Lefebvre, W., Harangozo, S. A.,
Adams, N., Scambos, T. A., Pook, M. J., \& Fowler, C. (2008). West
Antarctic Peninsula sea ice in 2005: Extreme ice compaction and ice edge
retreat due to strong anomaly with respect to climate. \emph{Journal of
Geophysical Research: Oceans}, \emph{113}(2), 1--23.
\url{https://doi.org/10.1029/2007JC004239}

\leavevmode\hypertarget{ref-Mayewski2017}{}%
Mayewski, P. A., Carleton, A. M., Birkel, S. D., Dixon, D., Kurbatov, A.
V., Korotkikh, E., McConnell, J., Curran, M., Cole-Dai, J., Jiang, S.,
Plummer, C., Vance, T., Maasch, K. A., Sneed, S. B., \& Handley, M.
(2017). Ice core and climate reanalysis analogs to predict Antarctic and
Southern Hemisphere climate changes. \emph{Quaternary Science Reviews},
\emph{155}, 50--66.
\url{https://doi.org/10.1016/j.quascirev.2016.11.017}

\leavevmode\hypertarget{ref-Morley2020}{}%
Morley, S. A., Abele, D., Barnes, D. K. A., Cárdenas, C. A., Cotté, C.,
Gutt, J., Henley, S. F., Höfer, J., Hughes, K. A., Martin, S. M.,
Moffat, C., Raphael, M., Stammerjohn, S. E., Suckling, C. C., Tulloch,
V. J. D., Waller, C. L., \& Constable, A. J. (2020). Global Drivers on
Southern Ocean Ecosystems: Changing Physical Environments and
Anthropogenic Pressures in an Earth System. \emph{Frontiers in Marine
Science}, \emph{7}(December), 1--24.
\url{https://doi.org/10.3389/fmars.2020.547188}

\leavevmode\hypertarget{ref-Newman2016}{}%
Newman, M., Alexander, M. A., Ault, T. R., Cobb, K. M., Deser, C., Di
Lorenzo, E., Mantua, N. J., Miller, A. J., Minobe, S., Nakamura, H.,
Schneider, N., Vimont, D. J., Phillips, A. S., Scott, J. D., \& Smith,
C. A. (2016). The Pacific decadal oscillation, revisited. \emph{Journal
of Climate}, \emph{29}(12), 4399--4427.
\url{https://doi.org/10.1175/JCLI-D-15-0508.1}

\leavevmode\hypertarget{ref-Nicolas2014}{}%
Nicolas, J. P., \& Bromwich, D. H. (2014). New reconstruction of
antarctic near-surface temperatures: Multidecadal trends and reliability
of global reanalyses. \emph{Journal of Climate}, \emph{27}(21),
8070--8093. \url{https://doi.org/10.1175/JCLI-D-13-00733.1}

\leavevmode\hypertarget{ref-ILAIA2021}{}%
Piñones, A. (2021). ILAIA ADVANCES IN CHILE ANANTARTIC SCi ENcE
n7--2021. \emph{Antartic Krill in a Time of Change}.

\leavevmode\hypertarget{ref-Pinones2016}{}%
Piñones, A., \& Fedorov, A. V. (2016). Projected changes of Antarctic
krill habitat by the end of the 21st century. \emph{Geophysical Research
Letters}, \emph{43}(16), 8580--8589.
\url{https://doi.org/10.1002/2016GL069656}

\leavevmode\hypertarget{ref-Rohli2018}{}%
Rohli, R. V., \& Vega, A. J. (2018). General Circulation and Secondary
Circulations. \emph{Climatology}, 131--154.

\leavevmode\hypertarget{ref-Scott2019}{}%
Scott, M. (2019). \emph{Antarctica is colder than the Arctic, but it's
still losing ice}.
\url{https://www.climate.gov/news-features/features/antarctica-colder-arctic-it's-still-losing-ice}

\leavevmode\hypertarget{ref-Scott2020}{}%
Scott, M. (2020). \emph{Understanding climate: Antarctic sea ice extent}
{[}PhD thesis{]}.
\url{https://www.climate.gov/news-features/understanding-climate/understanding-climate-antarctic-sea-ice-extent}

\leavevmode\hypertarget{ref-Stammerjohn2008a}{}%
Stammerjohn, S. E., Martinson, D. G., Smith, R. C., \& Iannuzzi, R. A.
(2008). Sea ice in the western Antarctic Peninsula region:
Spatio-temporal variability from ecological and climate change
perspectives. \emph{Deep-Sea Research Part II: Topical Studies in
Oceanography}, \emph{55}(18-19), 2041--2058.
\url{https://doi.org/10.1016/j.dsr2.2008.04.026}

\leavevmode\hypertarget{ref-Stammerjohn2008}{}%
Stammerjohn, S. E., Martinson, D. G., Smith, R. C., Yuan, X., \& Rind,
D. (2008). Trends in Antarctic annual sea ice retreat and advance and
their relation to El Niño-Southern Oscillation and Southern Annular Mode
variability. \emph{Journal of Geophysical Research: Oceans},
\emph{113}(3), 1--20. \url{https://doi.org/10.1029/2007jc004269}

\leavevmode\hypertarget{ref-Sylvester2021}{}%
Sylvester, Z. T., Long, M. C., \& Brooks, C. M. (2021). Detecting
Climate Signals in Southern Ocean Krill Growth Habitat. \emph{Frontiers
in Marine Science}, \emph{8}(June), 1--15.
\url{https://doi.org/10.3389/fmars.2021.669508}

\leavevmode\hypertarget{ref-Turner2004}{}%
Turner, J. (2004). The El Niño-Southern Oscillation and Antarctica.
\emph{International Journal of Climatology}, \emph{24}(1), 1--31.
\url{https://doi.org/10.1002/joc.965}

\leavevmode\hypertarget{ref-Turner2005}{}%
Turner, J., Colwell, S. R., Marshall, G. J., Lachlan-Cope, T. A.,
Carleton, A. M., Jones, P. D., Lagun, V., Reid, P. A., \& Iagovkina, S.
(2005). Antarctic climate change during the last 50 years.
\emph{International Journal of Climatology}, \emph{25}(3), 279--294.
\url{https://doi.org/10.1002/joc.1130}

\leavevmode\hypertarget{ref-Turner2016}{}%
Turner, J., Lu, H., White, I., King, J. C., Phillips, T., Hosking, J.
S., Bracegirdle, T. J., Marshall, G. J., Mulvaney, R., \& Deb, P.
(2016). Absence of 21st century warming on Antarctic Peninsula
consistent with natural variability. \emph{Nature}, \emph{535}(7612),
411--415. \url{https://doi.org/10.1038/nature18645}

\leavevmode\hypertarget{ref-Vorrath2020}{}%
Vorrath, M. E., Müller, J., Rebolledo, L., Cárdenas, P., Shi, X., Esper,
O., Opel, T., Geibert, W., Munoz, P., Haas, C., Kuhn, G., Lange, C. B.,
Lohmann, G., \& Mollenhauer, G. (2020). Sea ice dynamics in the
Bransfield Strait, Antarctic Peninsula, during the past 240 years: A
multi-proxy intercomparison study. \emph{Climate of the Past},
\emph{16}(6), 2459--2483. \url{https://doi.org/10.5194/cp-16-2459-2020}

\leavevmode\hypertarget{ref-Walsh2020}{}%
Walsh, J., Reiss, C. S., \& Watters, G. M. (2020). Flexibility in
antarctic krill euphausia superba decouples diet and recruitment from
overwinter sea-ice conditions in the northern antarctic peninsula.
\emph{Marine Ecology Progress Series}, \emph{642}, 1--19.
\url{https://doi.org/10.3354/meps13325}

\leavevmode\hypertarget{ref-Yuan2004}{}%
Yuan, X. (2004). ENSO-related impacts on Antarctic sea ice: A synthesis
of phenomenon and mechanisms. \emph{Antarctic Science}, \emph{16}(4),
415--425. \url{https://doi.org/10.1017/S0954102004002238}
\end{cslreferences}


\newpage
\begin{figure}

\vspace*{1cm}
\includegraphics[]{images/Fig11.jpg}

\end{figure}

\vspace*{0.5cm}

\begin{center}
\emph{Concentración de hielo marino en la Antártida el 28 de septiembre de 2020, día en que el hielo alcanzó su máxima extensión invernal. En comparación con las condiciones medias de 1981 a 2010 (línea amarilla), el máximo más reciente estuvo ligeramente por debajo del promedio (ex. www.climate.gov)}


\end{center}


\end{document}
